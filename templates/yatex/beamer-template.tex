\documentclass[aspectratio=169,cjk]{beamer} %オプション handout で印刷用
\usepackage{appendixnumberbeamer} %appendixを総ページ番号に含めない
\usepackage{graphicx}
\usepackage{tabularray}
\UseTblrLibrary{booktabs}
\usepackage[no-math,deluxe]{luatexja-preset}
\renewcommand{\kanjifamilydefault}{\gtdefault}
\renewcommand{\emph}[1]{{\upshape\bfseries #1}}
\usetheme{Boadilla}
% \usetheme{metropolis}
\setbeamertemplate{navigation symbols}{}
\setbeamertemplate{enumerate item}[default]
\setbeamertemplate{itemize item}[default]
\setbeamertemplate{section in toc}[sections numbered]
\setbeamertemplate{subsection in toc}{\hspace{1.2em}{\rule[0.3ex]{3pt}{3pt}}~\inserttocsubsection\par}

\title[短縮形]{ドキュメントタイトル}
\subtitle{サブタイトル}
\author{著者}
\institute[所属短縮系]{所属}
\date[短縮形]{\today}

\begin{document}

\begin{frame}[plain]
 \titlepage
\end{frame}

\begin{frame}[t] % t のほかに c, b が指定可能
 \frametitle{発表の流れ}
 \framesubtitle{目次のタイトル}
 \tableofcontents % 目次に載るのは {,サブ}セクション名。フレームタイトルは関係ない
\end{frame}

\section[セクション名短縮系1]{セクション名1}
\subsection[セクション名短縮系1-1]{サブセクション1}

\begin{frame}
 \frametitle{ブロックのサンプル}
 \framesubtitle{サブタイトル}

 \begin{block}{見出し}
  文字の\structure{強調}

  \begin{structureenv}
   ブロック単位の強調
  \end{structureenv}
 \end{block}

 \begin{alertblock}{警告ブロック}
  文字の\alert{警告}

  \begin{alertenv}
   ブロック単位の警告
  \end{alertenv}
 \end{alertblock}

 \begin{exampleblock}{例示ブロック}
  ブロック
 \end{exampleblock}

\end{frame}

\subsection[セクション名短縮系1-2]{サブセクション2}
\begin{frame}
 \frametitle{箇条書きのサンプル}
 \framesubtitle{サブタイトル}
 \begin{itemize}
  \item アイテム1
  \item アイテム2
 \end{itemize}
 \pause
 \begin{enumerate}
  \item アイテム1
  \item アイテム2
 \end{enumerate}
 \pause
 \begin{description}
  \item[項目1] \mbox{} \\
			 説明1
  \item[項目2] \mbox{} \\
			 説明2
 \end{description}
\end{frame}

\section[セクション名短縮系2]{セクション名2}
\begin{frame}
 \frametitle{2段組サンプル}
 \framesubtitle{テーブルと図のサンプル}
 \begin{columns}[onlytextwidth]
  \begin{column}[T]{0.45\hsize} % T, t c bがある
   \begin{center}
	\begin{tblr}{llX} \toprule
	 \SetCell{c} 項目A & 項目B & 項目C \\ \midrule
	 A & B & C \\ \hline
	 A & B & C \\ \hline
	 A & B & C \\ \bottomrule
	\end{tblr}
   \end{center}
  \end{column}
  \begin{column}[t]{0.45\hsize} % T, t c bがある
   \begin{center}
	\includegraphics[width=\columnwidth,height=5cm,keepaspectratio,scale=0.8]{sample.eps}
   \end{center}
  \end{column}
 \end{columns}

\end{frame}

\section*{セクションX}

\appendix
% \section{\appendixname}
\begin{frame}[fragile] % \verbやverbatimを使う場合
 \frametitle{予備のフレーム}
 \framesubtitle{サブタイトル}
 \verb|\appendix|以降のsectionは目次に反映されない。

 \verb|\section*|は目次に反映されない。

\end{frame}

\end{document}